%%
%% Copyright 2015 Alexander Chepurnoy
%%

\documentclass[number,preprint,review]{elsarticle}

%% Use the options 1p,twocolumn; 3p; 3p,twocolumn; 5p; or 5p,twocolumn
%% for a journal layout:
%% \documentclass[final,1p,times]{elsarticle}
%% \documentclass[final,1p,times,twocolumn]{elsarticle}
%% \documentclass[final,3p,times]{elsarticle}
%% \documentclass[final,3p,times,twocolumn]{elsarticle}
%% \documentclass[final,5p,times]{elsarticle}
%% \documentclass[final,5p,times,twocolumn]{elsarticle}

%% For including figures, graphicx.sty has been loaded in
%% elsarticle.cls. If you prefer to use the old commands
%% please give \usepackage{epsfig}

%% The amssymb package provides various useful mathematical symbols
\usepackage{amssymb}
%% The amsthm package provides extended theorem environments
\usepackage{amsthm}

%% for url reference
\usepackage{hyperref}


%% The lineno packages adds line numbers. Start line numbering with
%% \begin{linenumbers}, end it with \end{linenumbers}. Or switch it on
%% for the whole article with \linenumbers.
%% \usepackage{lineno}

\journal{Ledger Journal}

\begin{document}

\begin{frontmatter}

%% Title, authors and addresses

%% use the tnoteref command within \title for footnotes;
%% use the tnotetext command for theassociated footnote;
%% use the fnref command within \author or \address for footnotes;
%% use the fntext command for theassociated footnote;
%% use the corref command within \author for corresponding author footnotes;
%% use the cortext command for theassociated footnote;
%% use the ead command for the email address,
%% and the form \ead[url] for the home page:
%% \title{Title\tnoteref{label1}}
%% \tnotetext[label1]{}
%% \author{Name\corref{cor1}\fnref{label2}}
%% \ead{email address}
%% \ead[url]{home page}
%% \fntext[label2]{}
%% \cortext[cor1]{}
%% \address{Address\fnref{label3}}
%% \fntext[label3]{}

\title{Hybridicum: A Hybrid Blockchain Consensus Protocol}


%% use optional labels to link authors explicitly to addresses:
%% \author[label1,label2]{}
%% \address[label1]{}
%% \address[label2]{}

\author{Alexander Chepurnoy}
\ead{kushti@protonmail.ch}

\address{}

\begin{abstract}

\end{abstract}

\begin{keyword}
Blockchain \sep Decentralized consensus \sep Proof-of-stake \sep Peer-to-peer networks \sep Proof-of-Work
\end{keyword}

\end{frontmatter}


\section{Introduction}
\label{intr_section}




\textbf{TODO: terminology and notations}


\section{Interactive Proof-of-Stake}



\section{Properties For A New Blockchain Consensus Protocol}


We formulate following properties of a new blockchain consensus protocol we wish to achieve.

\begin{enumerate}

\item Block generators in Proof-of-Work are incentivized to contribute to one chain only. In opposite, block generators in Proof-of-Stake are incentivized to contribute to as many chains as possible. We would like to get chain selection from a tree enforced in an elegant way.

\item Proof-of-Stake provides an incentive to run a fullnode.

\item We are willing to spread transaction fees amongst few parties. Otherwise a block generator could include her transactions for free. There are some attacks linked with that, for example \cite{???}

\end{enumerate}

\section{}




\section{The Hybrid Consensus Protocol}



With Interactive Proof-of-Stake miners still have incentive to contribute to a multiple chains. Also grinding attacks are probably possible(???).


To overcome the shortcomings aforementioned, the idea of ours is to use both Proof-of-Work and Interactive Proof-of-Stake to secure a blockchain system. 

We use Proof-of-Stake blocks(\textit{containers}) to carry transactions. We use a Proof-of-Work block(\textit{deciders}) as a random beacon\cite{???} and also as a voting mechanism to enforce some chain choosing from a tree.

A Proof-of-Work block contains some \(puz\) value\cite{???}. Consider \(hash(puz \cup pk)\) has length of 30 bytes at least, so we take first 30 bytes of it and consider those bytes as 10*3 \(m\) values to generate tickets for next 10 blocks container blocks. For example, for first 3 bytes up to three tickets for a first block could be generated. 

So if online miners set is known next 10 container blocks generators are also known, but network interaction is needed to generate those blocks. 


\subsection*{Referencing rules}

\begin{enumerate}
\item Each container block refers to a previous container block and a parent decider block as well. 
\item Each decider block contains references to a parent decider block and last seen container block. Last seen container block must be a descendant of a parent's last seen container block.  
\end{enumerate}

\subsection*{Rewarding rules}
\begin{enumerate}
\item For a container block, transaction fees are divided equally amongst ticket generators.
\item For a decider block, reward is \(C*\frac{\delta s_{cur}}{\delta s_{prev}} \), where \(\delta s_{prev}\) is blockscore 
increasement fixed by previous decider and \(\delta s_{cur}\) is current increasement.
\end{enumerate}



\section{The Protocol Analysis}


\section{Possible Improvements}
\label{prot_improv}



\section{Related Work}


\section{Further Work}


\section{Conclusion}
\label{conclusion}


%% If you have bibdatabase file and want bibtex to generate the
%% bibitems, please use
%%
%%  \bibliographystyle{elsarticle-num}
%%  \bibliography{<your bibdatabase>}

%% else use the following coding to input the bibitems directly in the
%% TeX file.

\section*{Acknowledgement}

\section*{Author Contributions}

\section*{Conflict of Interest}

I have no any conflict of interest to declare. 




\section*{References}

\bibliographystyle{elsarticle-num}
\bibliography{sources.bib}


\end{document}