%%
%% Copyright 2016 Dmitry Meshkov
%%

\documentclass[number,preprint,review]{elsarticle}

%% Use the options 1p,twocolumn; 3p; 3p,twocolumn; 5p; or 5p,twocolumn
%% for a journal layout:
%% \documentclass[final,1p,times]{elsarticle}
%% \documentclass[final,1p,times,twocolumn]{elsarticle}
%% \documentclass[final,3p,times]{elsarticle}
%% \documentclass[final,3p,times,twocolumn]{elsarticle}
%% \documentclass[final,5p,times]{elsarticle}
%% \documentclass[final,5p,times,twocolumn]{elsarticle}

%% For including figures, graphicx.sty has been loaded in
%% elsarticle.cls. If you prefer to use the old commands
%% please give \usepackage{epsfig}

%% The amssymb package provides various useful mathematical symbols
\usepackage{amssymb}
%% The amsthm package provides extended theorem environments
\usepackage{amsthm}

%% for url reference
\usepackage{hyperref}


%% The lineno packages adds line numbers. Start line numbering with
%% \begin{linenumbers}, end it with \end{linenumbers}. Or switch it on
%% for the whole article with \linenumbers.
%% \usepackage{lineno}

\journal{??????????????????}

\begin{document}

\begin{frontmatter}

%% Title, authors and addresses

%% use the tnoteref command within \title for footnotes;
%% use the tnotetext command for theassociated footnote;
%% use the fnref command within \author or \address for footnotes;
%% use the fntext command for theassociated footnote;
%% use the corref command within \author for corresponding author footnotes;
%% use the cortext command for theassociated footnote;
%% use the ead command for the email address,
%% and the form \ead[url] for the home page:
%% \title{Title\tnoteref{label1}}
%% \tnotetext[label1]{}
%% \author{Name\corref{cor1}\fnref{label2}}
%% \ead{email address}
%% \ead[url]{home page}
%% \fntext[label2]{}
%% \cortext[cor1]{}
%% \address{Address\fnref{label3}}
%% \fntext[label3]{}

\title{Difficulty Control for Blockchain Systems}


%% use optional labels to link authors explicitly to addresses:
%% \author[label1,label2]{}
%% \address[label1]{}
%% \address[label2]{}

\author[iohk]{Dmitry Meshkov}
\ead{dmitry.meshkov@iohk.io}

\author[iohk]{Alexander Chepurnoy}


\address[iohk]{IOHK Research}

\begin{abstract}
\textbf{TODO}
\end{abstract}

\begin{keyword}
Blockchain \sep Decentralized consensus \sep Peer-to-peer networks \sep Proof-of-Work
\end{keyword}

\end{frontmatter}


\section{Introduction}

Blockchain systems have attracted a growing amount of interest from various communities after publication of Bitcoin whitepaper \cite{Nakamoto2008} in 2008.
Bitcoin security relies on the distributed protocol that maintains the blockchain, called mining, in which network nodes tries to solve computational puzzle.
Other blockchain systems may relies on different computational puzzles [??] or even on virtual mining [??], while all of them use some algorithm that changes puzzle difficulty dynamically.
This algorithm for retargeting the difficulty is required to make blockchain system predictable and fix latency between blocks.

Fixed latency between blocks is important for several reasons.
Too often blocks leads to situation, when for a lot of miners block propagation time become bigger, then latency between blocks.
This leads to significant increasing of number of blockchain forks that complicates the consensus\cite{decker2013information} and reduce effective hash rate in blockchain system.
On the other hand, increasing of the latency between blocks leads to decreasing of the network throughput\cite{miller2016} and may be critical for high-loaded blockchain systems like bitcoin, where blocks are already 70\% full today\cite{armstrong2016}.
Increasing latency by 50% in bitcoin network will mean, that some transactions will be never included into blockchain.
Moreover this will lead to infinite growth of unconfirmed transactions pool, meaning it is likely that most bitcoin transactions will not even relay, much less confirm.

Most of blockchain systems relies on quite a naive difficulty retargeting algorithm, that assumes, that total computational power, involved in mining process, don't change over time.
Using more complicated retargeting algorithms with incorrect assumptions\cite{andruiman2014} may lead to incorrect time interval between blocks even for simple case of constant hash rate as, for example, observed in Nxt, where observed mean time between blocks is ~2 times bigger, then expected in whitepaper\cite{nxt}. Moreover, too ofter difficulty recalculation leads to wide distribution of time intervals between blocks and makes blockchain system unpredictable\cite{andruiman2014}.
Varying network computational power makes this algorithms inefficient for difficulty recalculation, e.g. continuous growth of computational power leads to decreasing mean latency between blocks and average block time in Bitcoin network is ~1.07 times lower, then expected.
Noteworthy, that exponential growth of computational power, which is the situation observed in practice in accordance with Moore’s law\cite{moore2006cramming}, is the absolutely worst case (regarding the maximal block rate) possible for Bitcoin’s difficulty retargeting algorithm\cite{kraft2015difficulty}.

Original Bitcoin white-paper, states that the security of the system is guaranteed as long as there is no attacker in possession of half or more of the total computational power used to maintain the system {Nakamoto2008}.
However difficulty is not constant, and can be manipulated by the attacker.
The Difficulty Raising Attack, introduced in \cite{bahack2013theoretical}, enables the attacker to discard n-depth block, for any n and any attacker hash power, with probability 1 if he is willing to wait enough time.
The fact that there is no way to determine whether a block have been computed on its declared time or not, have been used as part of other attacks \cite{timejacking2011, artforz2011}.

Thus difficulty recalculation is small, but important part of blockchain systems \textbf{TODO}

\section{Bitcoin Mining}


\section{The Difficulty Control Attack}


\section{An Improved Difficulty Control}

\section{Simulations}


\section{Conclusions}
\label{prot_improv}

\section*{References}

\bibliographystyle{elsarticle-num}
\bibliography{sources.bib}


\end{document}